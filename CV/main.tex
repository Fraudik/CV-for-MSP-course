\documentclass[10pt,a4paper]{altacv}

\geometry{left=1cm,right=9cm,marginparwidth=6.8cm,marginparsep=1.2cm,top=1cm,bottom=1cm}

\usepackage[default]{lato}
\usepackage[T1,T2A]{fontenc}
\usepackage[utf8]{inputenc}
\usepackage[english,russian]{babel}
\usepackage{csquotes}

\definecolor{VividPurple}{HTML}{008080}
\definecolor{SlateGrey}{HTML}{2E2E2E}
\definecolor{LightGrey}{HTML}{666666}
\colorlet{heading}{VividPurple}
\colorlet{accent}{VividPurple}
\colorlet{emphasis}{SlateGrey}
\colorlet{body}{LightGrey}

\renewcommand{\itemmarker}{{\small\textbullet}}
\renewcommand{\ratingmarker}{\faCircle}

\begin{document}
\name{Блинов Илья}
\tagline{}
\personalinfo{
  \email{iiblinov@edu.hse.ru}
  \location{Тула, Россия}
  \linebreak
  \github{https://github.com/Fraudik} 
}

\begin{fullwidth}
\makecvheader
\end{fullwidth}

\AtBeginEnvironment{itemize}{\small}

\cvsection[page1sidebar]{Обучение}
\cvevent{Программа «Компьютерные науки и анализ данных» \\ Направление «Прикладная математика и информатика» \\ Специализация «Распределенные системы»}{Факультет компьютерных наук НИУ ВШЭ }{2021 - настоящее время}{Москва, Россия}
\textsc{1 курс: GPA 9.26, 1-ое место в рейтинге программы} \\
\textsc{2 курс: GPA 9.4, 1-ое место в рейтинге программы} \\
\textsc{3 курс: GPA 9.5, 1-ое место в рейтинге программы} \\
\textsc{4 курс: в процессе} \\

\vspace{10pt}
\cvevent{Основное образование}{Лицей №2, инженерный класс}{2014 - 2021}{Тула, Россия}
\textsc{Выпустился с золотой медалью}
\\
\vspace{10pt}
\cvevent{Дополнительное образование}{Яндекс Лицей}{2018 - 2020}{Тула, Россия}
\textsc{Закончил с отличием (вошёл в топ выпускников проекта). \\ Номер сертификата: 2002 21215}

\cvsection{Опыт работы}

\item[] Работа над проектом «Система сбора данных психологических опросов» 
\begin{itemize}
    \item[-] Занимался дизайном системы и координировал работу команды, а также работал над серверной частью
    \item[-] Технологический стек: FastAPI, REST, Pydantic, SQLAlchemy, Poetry, Alembic, Docker, Nginx
    \item[-] С сентября 2022 по март 2023
\end{itemize}

\item[] \textbf{ООО Саппи Аналитикс}
\begin{itemize}
    \item Бэкенд-разработчик на Python
    \begin{itemize}
        \item[-] Технологический стек: Django, Graphene, Pandas, Celery, asyncio, PostgreSQL, Docker, Nginx
        \item[-] С мая 2023
    \end{itemize}
\end{itemize}

\item[] \textbf{НИУ Высшая школа экономики, Факультет компьютерных наук}
\begin{itemize}
    \item Ассистент на курсе «Программирование на Python»
    \begin{itemize}
        \item[-] Сентябрь 2022 ---- октябрь 2022
    \end{itemize}
    \item Ассистент на курсе «Программирование на С++»
    \begin{itemize}
        \item[-] Январь 2022 --- март 2022
        \item[-] Январь 2023 --- март 2023
    \end{itemize}
    \item Ассистент на курсе «Алгоритмы и структуры данных»
    \begin{itemize}
        \item[-] Сентябрь 2023 --- декабрь 2023
        \item[-] C сентября 2024
    \end{itemize}
\end{itemize}

\item[] \textbf{Летняя олимпиадная штука, МФТИ}
\begin{itemize}
    \item Ассистент на направлении «Информатика»
    \begin{itemize}
        \item[-] Конец июня 2024
    \end{itemize}
        \item Лектор на направлении «Информатика»
    \begin{itemize}
        \item[-] Конец июля 2024
    \end{itemize}
\end{itemize}

\clearpage


\end{document}
