\documentclass[10pt,a4paper]{altacv}

\geometry{left=1cm,right=9cm,marginparwidth=6.8cm,marginparsep=1.2cm,top=1cm,bottom=1cm}

\usepackage[default]{lato}
\usepackage[T1,T2A]{fontenc}
\usepackage[utf8]{inputenc}
\usepackage[english,russian]{babel}
\usepackage{csquotes}

\definecolor{VividPurple}{HTML}{008080}
\definecolor{SlateGrey}{HTML}{2E2E2E}
\definecolor{LightGrey}{HTML}{666666}
\colorlet{heading}{VividPurple}
\colorlet{accent}{VividPurple}
\colorlet{emphasis}{SlateGrey}
\colorlet{body}{LightGrey}

\renewcommand{\itemmarker}{{\small\textbullet}}
\renewcommand{\ratingmarker}{\faCircle}

\begin{document}
\name{Блинов Илья}
\tagline{}
\personalinfo{
  \email{iiblinov@edu.hse.ru}
  \location{Тула, Россия}
  \linebreak
  \github{https://github.com/Fraudik} 
}

\begin{fullwidth}
\makecvheader
\end{fullwidth}

\AtBeginEnvironment{itemize}{\small}

\cvsection[page1sidebar]{Обучение}
\cvevent{Программа «Компьютерные науки и анализ данных» \\ Направление «Прикладная математика и информатика»}{Факультет компьютерных наук НИУ ВШЭ }{2021 - настоящее время}{Москва, Россия}
\textsc{1 курс: GPA 9.26, 1-ое место в рейтинге программы} \\
\textsc{2 курс: текущий GPA 9.69, 1-ое место в рейтинге программы}
\\
\vspace{10pt}
\cvevent{Инженерный класс}{Лицей №2}{2014 - 2021}{Тула, Россия}
\textsc{Выпустился с золотой медалью}
\\
\vspace{10pt}
\cvevent{Дополнительное образование}{Яндекс Лицей}{2018 - 2020}{Тула, Россия}
\textsc{Закончил с отличием (вошёл в топ выпускников проекта). \\ Номер сертификата: 2002 21215}


\cvsection{Проекты}
\begin{itemize}
\item Создание группы микросервисов в команде в рамках проекта «Система сбора данных психологических опросов» \\
Находится на стадии разработки, в проекте ответственен за серверную часть с использованием FastAPI, консультирование в этой области и развертывание микросервисов с помощью Docker и Nginx. \\
\item Шаблон для серверной части на основе фреймворка FastAPI для проекта «Система сбора данных психологических опросов» \\
(https://github.com/Fraudik/FastAPI-template) \\
\item Небольшой сервис для размещения записей (блогов)
https://github.com/Fraudik/FastAPI\_blog \\
\item Автоматическая компиляция и публикация резюме с помощью Docker и GitHub Actions \\
(https://github.com/Fraudik/CV-with-auto-deployment) \\
\item  Консольное приложение на С++, позволяющее применять к изображениям в формате BMP различные фильтры \\
(https://github.com/Fraudik/BMP-Image-Processor) \\
\end{itemize}

\cvsection{Опыт работы}
\begin{itemize}
\item[] \textbf{НИУ Высшая школа экономики, Факультет компьютерных наук}
\begin{itemize}
    \item Ассистирование на курсе «Программирование на Python»
    \begin{itemize}
        \item[-] Набор «КНАД 2022» бакалаврской программы «Компьютерные науки и анализ данных»
        \item[-] Сентябрь 2022 - октябрь 2022
    \end{itemize}
    \item Ассистирование на курсе «Программирование на С++»
    \begin{itemize}
        \item[-] Набор «КНАД 2022» бакалаврской программы «Компьютерные науки и анализ данных»
        \item[-] Набор «ВСН 2022» бакалаврской программы «Вычислительные социальные науки»
        \item[-] Январь 2023 --- март 2023
    \end{itemize}
\end{itemize}
\end{itemize}


\clearpage


\end{document}